\documentclass[nofonts]{tufte-handout}

\usepackage{polyglossia}
\setdefaultlanguage{french}
\usepackage{booktabs}
\usepackage[locale=FR]{siunitx}
\usepackage{graphicx}
\usepackage{enumitem}
\usepackage{tikz}
\usepackage{amsmath}
\usetikzlibrary{positioning, calc}

\usepackage{fontspec}
\usepackage{ifluatex}
\setmainfont[Renderer=Basic, Numbers=OldStyle, Scale = 1.0]{TeX Gyre Pagella}
\setsansfont[Renderer=Basic, Scale=0.90]{TeX Gyre Heros}
\setmonofont[Renderer=Basic]{TeX Gyre Cursor}
\ifluatex
  \newcommand{\textls}[2][5]{%
    \begingroup\addfontfeatures{LetterSpace=#1}#2\endgroup
  }
  \renewcommand{\allcapsspacing}[1]{\textls[15]{#1}}
  \renewcommand{\smallcapsspacing}[1]{\textls[10]{#1}}
  \renewcommand{\allcaps}[1]{\textls[15]{\MakeTextUppercase{#1}}}
  \renewcommand{\smallcaps}[1]{\smallcapsspacing{\scshape\MakeTextLowercase{#1}}}
  \renewcommand{\textsc}[1]{\smallcapsspacing{\textsmallcaps{#1}}}
\fi

\geometry{
  left=1.5cm,
  top=2cm,
  textwidth=30pc,
  bottom=1.5cm,
}

\sisetup{
  mode=text,
  reset-text-family=false,
  reset-text-series=false,
  per-mode=symbol,
}

\newcommand{\F}{\boldsymbol{\vec{F}}}
\newcommand{\vv}{\boldsymbol{\vec{v}}}
\newcommand{\va}{\boldsymbol{\vec{a}}}


\title{Grille de correction des problèmes}
\author{Loïc Séguin-Charbonneau}
\date{203-NYB-05, Automne 2024}

\begin{document}

\maketitle

\section{Critères de correction et indicateurs}

\begin{fullwidth}
Les problèmes (problèmes intégrateurs et problèmes dans les examens) seront
corrigés selon les quatre critères dans le tableau ci-dessous. Pour chacun des
  critères, le tableau montre les critères de perfomance ministériels (CPM)
correspondant de même que des indicateurs qui attestent de la satisfaction du
critère.
Pour rappel, les critères de performance ministériels sont les suivants:
\begin{enumerate}[label=\alph*.]
  \item utilisation appropriée des concepts, des principes et des lois de l’électricité et du magnétisme;
  \item schématisation adéquate des situations physiques;
  \item représentations graphiques adaptées à la nature des phénomènes;
  \item justification des étapes retenues pour l’analyse des situations;
  \item application rigoureuse des concepts, les principes et les lois de l’électricité et du magnétisme;
  \item jugement critique des résultats;
  \item interprétation des limites des modèles;
\end{enumerate}

\vspace{\baselineskip}
La correction d'un critère est conditionnelle à la maîtrise des critères
précédents. Par exemple, la résolution mathématique d'un étudiant qui ne
réussit pas à trouver un principe de résolution judicieux ne sera pas
corrigée.


\vspace{\baselineskip}

\noindent\begin{tabular}{p{3cm}p{1cm}p{13cm}}
  \toprule
  Critère  &  CPM  &  Indicateurs   \\
  \midrule
  Schématisation juste
    &  b
    &  Effectuer un schéma de principe utile et clair qui
       \begin{itemize}
         \item présente les paramètres de la situation (forces, champs, dimensions, etc.);
         \item inclut les systèmes de coordonnées appropriés;
         \item présente les éléments relatifs à l'évolution de la situation (conditions initiales, finales, intermédiaires).
       \end{itemize}
       \\\midrule
  Principe de résolution judicieux et explicite
    & a, d, f, g
    & \vspace{-\baselineskip}\begin{itemize}
        \item Présenter les grandes lignes de la résolution en un enchaînement logique et clair
        \item Associer la situation aux lois et principes appropriés
        \item Justifier chaque étape de la résolution par des lois, principes
          phyiques, règles mathématiques ou raisonnements logiques appropriés et
          justes
        \item Noter les contraintes et limites applicables à la situation
        \item Relever des résultats impossibles ou aberrants
      \end{itemize}
      \\\midrule
  Résolution mathématique exacte
    &  e, c
    & \vspace{-\baselineskip}\begin{itemize}
        \item Associer avec justesse les variables aux quantités
        \item Présenter une démarche explicite, complète et claire
        \item Obtenir les résultats exacts.
        \item Illustrer le comportement des variables par un graphique exact
      \end{itemize}  \\\midrule
  Respect des normes et conventions
    &  e
    & \vspace{-\baselineskip}\begin{itemize}
        \item Respecter la syntaxe mathématique (e.x.: la notation scalaire/vectorielle)
        \item Effectuer une analyse dimensionnelle pour s'assurer de la cohérence des unités
      \end{itemize}  \\
  \bottomrule
\end{tabular}
\end{fullwidth}

\section{Rubrique d'évaluation}

Chaque critère sera évalué sur une échelle à quatre niveaux : \textbf{E}xcellent,
\textbf{S}atisfaisant, en p\textbf{R}ogression,
\textbf{F}ragmentaire\cite{stutzman2004emrf}. La signification de chaque niveau
est décrite dans la figure ci-dessous.

\begin{figure*}
\begin{tikzpicture}[
    node distance=8mm and 0mm,
    outer sep=4pt,
    box/.style={
      rectangle,
      very thick,
      draw=black!50,
      text width=4cm,
      align=center
    },
    terminal/.style={
      rectangle,
      rounded corners=3mm,
      very thick,
      draw=black!50,
      fill=black!8,
      text width=3.6cm,
      align=center
    },
    >=stealth
  ]
  \node (top) [box] {
    Est-ce que le travail démontre une compréhension des concepts et
    répond aux attentes?
  };
  \node (complet) [box, below left=of top] {Est-il complet et bien communiqué?};
  \node (partiel) [box, below right=of top] {Y a-t-il des signes d'une compréhension partielle?};
  \node (excellent) [terminal, below left=of complet, xshift=2cm] {
    \textbf{Excellent}. Satisfait ou dépasse toutes les attentes. La communication est
    claire et complète. La maîtrise des concepts est évidente. Aucune erreur
    non triviale n'est présente.
  };
  \node (sat) [terminal, below right=of complet, xshift=-2cm] {
    \textbf{Satisfaisant}. Les attentes sont atteintes et la compréhension est évidente.
    Quelques révisions ou améliorations sont requises, mais aucune erreur
    significative n'est présente. Aucun enseignement supplémentaire n'est requis.
  };
  \node (prog) [terminal, below left=of partiel, xshift=2cm] {
    \textbf{En progression}. Une compréhension partielle est évidente,
    mais des lacunes importantes demeurent. Requiert plus de travail, une
    révision importante ou des explications améliorées.
  };
  \node (frag) [terminal, below right=of partiel, xshift=-2cm] {
    \textbf{Fragmentaire}. Trop peu d'information est présente pour pouvoir évaluer
    adéquatement si les concepts sont compris. Le travail contient des
    omissions importantes ou il y a trop de problèmes pour porter un jugement
    global.
  };
  \draw[->] (top.south) -- ++(0, -0.4) -| node[left, xshift=-5pt] {oui} (complet.north);
  \draw[->] (top.south) -- ++(0, -0.4) -| node[right, xshift=5pt] {non} (partiel.north);
  \draw[->] (complet.south) -- ++(0, -0.4) -| node[left, xshift=-5pt] {oui} (excellent.north);
  \draw[->] (complet.south) -- ++(0, -0.4) -| node[right, xshift=5pt] {non} (sat.north);
  \draw[->] (partiel.south) -- ++(0, -0.4) -| node[left, xshift=-5pt] {oui} (prog.north);
  \draw[->] (partiel.south) -- ++(0, -0.4) -| node[right, xshift=5pt] {non} (frag.north);
  \draw[densely dashed] ($(top) + (0, -2.5)$) -- ++(0, -7.5) node[below] {seuil};
\end{tikzpicture}
\end{figure*}

\section{Grille de correction}

\begin{tabular}{ll}
  \toprule
  Critère  &  Niveau  \\
  \midrule
  Schématisation juste  &  E\quad\quad S\quad\quad R\quad\quad F\quad\quad \\
  Principe de résolution judicieux et explicite  &  E\quad\quad S\quad\quad R\quad\quad F\quad\quad \\
  Résolution mathématique exacte  &  E\quad\quad S\quad\quad R\quad\quad F\quad\quad \\
  Respect des normes et conventions  &  E\quad\quad S\quad\quad R\quad\quad F\quad\quad \\
  \bottomrule
\end{tabular}

\vspace{\baselineskip}
Pour produire une note chiffrée à partir de la grille, une moyenne pondérée est
effectuée de la façon suivante. La pondération respective de chaque critère est
de \qty{20}{\percent}, \qty{30}{\percent}, \qty{30}{\percent} et
\qty{20}{\percent}. Le pourcentage associé à chaque niveau est E:
\qty{100}{\percent}, S: \qty{75}{\percent}, R: \qty{40}{\percent}, F:
\qty{0}{\percent}. Par exemple, une personne étudiante qui aurait EESS aura une
note de \qty{87.5}{\percent}. Une personne étudiante avec une évaluation SRRF
aura une note de \qty{39}{\percent}.

\nobibliography{../elecmag}
\bibliographystyle{plainnat}

\end{document}

\documentclass[letterpaper, DIV=11]{scrartcl}

\usepackage{polyglossia}
\setdefaultlanguage{french}

\usepackage[backend=biber]{biblatex}
\addbibresource{../elecmag.bib}

\usepackage{fontspec}
\setmainfont[Renderer=Basic, Numbers=OldStyle, Scale = 1.0]{TeX Gyre Pagella}
\setsansfont[Renderer=Basic, Scale=0.90]{TeX Gyre Heros}
\setmonofont[Renderer=Basic]{TeX Gyre Cursor}

\setkomafont{title}{
  \normalfont
  \itshape
  \rmfamily
}
\addtokomafont{author}{
  \itshape
}
\addtokomafont{date}{
  \itshape
}
\setkomafont{section}{
  \normalfont
  \itshape
  \rmfamily
  \Large
}

\usepackage{booktabs}
\usepackage{graphicx}
\usepackage{enumitem}
\usepackage{tikz}
\usepackage{amsmath}
\usepackage[locale=FR]{siunitx}
\usepackage{hyperref}

\sisetup{
  mode=text,
  reset-text-family=false,
  reset-text-series=false,
  per-mode=symbol,
  uncertainty-mode=separate,
}

\newcommand{\F}{\boldsymbol{\vec{F}}}
\newcommand{\vv}{\boldsymbol{\vec{v}}}
\newcommand{\va}{\boldsymbol{\vec{a}}}


\title{Miniguide méthodologique}
\author{Loïc Séguin-Charbonneau}
\date{}

\begin{document}

\maketitle

Ce miniguide est un aide mémoire très condensé. Si vous avez besoin
d'explications détaillées ou de plus de détails sur les bonnes pratiques
méthodologiques, vous pouvez consulter une des nombreuses références sur le
sujet (par exemple,
\cite{boisclair_page_2014,debellefeuille_ite,jcgm_2008,laflamme_2018}). Pour des
définitions précises de tous les termes de métrologie, vous pouvez consulter le
Vocabulaire international de métrologie \cite{vim}.

\section*{Règles d'écriture du résultat d'un mesurage}

Lorsqu'on écrit un résultat de mesurage, que ce soit un résultat de mesurage
obtenu directement avec un instrument de mesure ou obtenu indirectement à
partir de calculs, on écrit la valeur mesurée, l'incertitude et l'unité en
respectant les règles suivantes:
\begin{enumerate}
  \item l'incertitude s'écrit avec un ou deux chiffres significatifs;
  \item le nombre de positions décimales de la valeur mesurée est le même que
    celui de l'incertitude;
  \item la valeur mesurée est suivie du signe $\pm$, puis de l'incertitude, le
    tout entre parenthèses et suivi de l'unité;
  \item lorsque nécessaire, on utilise la notation scientifique en prenant soin
    d'utiliser la même puissance de 10 pour la valeur mesurée et l'incertitude.
\end{enumerate}

\begin{table}
  \centering
  \begin{tabular}{lll}
    \toprule
    Exemple correct    & Exemple incorrect  &  Règle violée  \\
    \midrule
    \qty{23.4123(18)}{\milli\ampere}  &  \qty{23.41231(187)}{\milli\ampere}  &  1  \\
    \qty{123.7(8)}{\micro\farad}  &  $\left(\num{123.7} \pm \num{0.08}\right)\unit{\micro\farad}$  &  2  \\
    \qty{8.79(3)}{\volt}  &  $\num{8.79} \pm \num{0.03} \unit{\volt}$  &  3  \\
    \qty{4.284(2)e-5}{\meter}  &  $\left(\num{4.284e-5} \pm
    \num{2e-8}\right)\unit{\meter}$  &  4\\
    \bottomrule
  \end{tabular}
  \caption{Exemples d'utilisation des règles d'écriture pour les résultats de mesurage.}
\end{table}



\section*{Évaluation et propagation des incertitudes}

Lorsqu'on effectue un mesurage, il est impossible de connaître exactement la
valeur du mesurage \cite{jcgm_2008}. Il faut évaluer l'incertitude en additionnant
l'incertitude instrumentale et l'incertitude liée au contexte.
L'incertitude instrumentale peut être déterminée à partir de règles empiriques
(par exemple, prendre la moitié de la plus petite division pour une lecture sur
un instrument graduée) ou être fournie par le fabricant (par exemple, le
fabricant d'un multimètre numérique donne un tableau détaillé pour le calcul des
incertitudes instrumentales).

On évalue les incertitudes liées au contexte en tenant compte de la méthode de
mesure. Lorsqu'il fait cette évaluation, l'expérimentateur doit être confiant
que la valeur du mesurande se situe dans l'intervalle définit par l'incertitude.
Par exemple, pour un instrument à affichage numérique, si la valeur sur
l'affichage fluctue, on peut attribuer une incertitude liée au contexte de la
moitié de l'écart entre les valeurs extrêmes.

Dans des cas simples, on peut calculer l'incertitude sur un résultat de mesurage
composé en utilisant les règles résumées dans le
tableau~\ref{tab:regles_simples}. Ces règles simples surestiment généralement
l'incertitude composée, mais elles donnent le bon ordre de grandeur.

\begin{table}
  \renewcommand{\arraystretch}{1.5}
  \centering
  \begin{tabular}{llll}
    \toprule
    Opération   & &  Valeur &  Incertitude  \\
    \midrule
    Somme  &  $R = A + B$
      &  $\tilde{R} = \tilde{A} + \tilde{B}$
      & $\Delta R = \Delta A + \Delta B$ \\
    Différence  &  $R = A - B$
      &  $\tilde{R} = \tilde{A} - \tilde{B}$
      & $\Delta R = \Delta A + \Delta B$ \\
    Produit  &   $R = A \times B$
      &  $\tilde{R} = \tilde{A} \times \tilde{B}$
      & $\Delta R = \left|\tilde{R}\right| \left(
          \displaystyle{\frac{\Delta A}{\left|\tilde{A}\right|}} + \displaystyle{\frac{\Delta
          B}{\left|\tilde{B}\right|}}
        \right)$ \\
    Quotient  &  $R = A / B$
      &  $\tilde{R} = \tilde{A} / \tilde{B}$
      & $\Delta R = \left|\tilde{R}\right| \left(
          \displaystyle{\frac{\Delta A}{\left|\tilde{A}\right|}} + \displaystyle{\frac{\Delta B}{\left|\tilde{B}\right|}}
        \right)$ \\
    Exposant  &  $R = A^n$
      &  $\tilde{R} = \tilde{A}^n$
      & $\Delta R = \left|\tilde{R}\right|
      \left(\left|n\right|\displaystyle{\frac{\Delta A}{\left|\tilde{A}\right|}}\right)$ \\
      \bottomrule
  \end{tabular}
  \caption{Règles simples de calcul d'incertitude pour des variables
  indépendantes $A = \tilde{A} \pm \Delta A$ et $B = \tilde{B} \pm \Delta B$.}
  \label{tab:regles_simples}
\end{table}


\section*{Présentation des tableaux}

Un tableau doit présenter les données de façon claire, concise et complète. Un
bon tableau maximise la quantité d'information fournie en minimisant l'encre
utilisée. Les colonnes doivent être clairement identifiées, les unités et les
incertitudes doivent être présentes, les données doivent être écrites en
respectant les règles d'écriture. La légende accompagnant le tableau doit
décrire son contenu suffisamment clairement pour qu'un lecteur comprenne la
nature des données qui y sont présentées. Les tableaux doivent être numérotés
pour pouvoir y faire facilement référence dans le texte.

Par exemple, considérons un laboratoire où on aurait mesuré la grandeur du
champ électrique produit par une charge ponctuelle à différentes distances de
la charge. L'expression théorique du champ électrique est
\begin{equation}
  \vec{E} = \frac{kq}{r^2}\vec{u}_r
  \label{eq:champ_ponctuel}
\end{equation}
Le tableau \ref{tab:data_er2} montre les différentes mesures obtenues lors de
cette expérience.

\begin{table}
  \renewcommand{\arraystretch}{1.5}
  \centering
  \begin{tabular}{SS}
    \toprule
    {$r$ (\unit{\meter})} &  {$E$ (\unit{\kilo\newton\per\coulomb})}  \\
    \midrule
        0.201(8)   &          9.82(3)  \\
        0.233(9)   &          7.04(3)  \\
        0.262(9)   &          5.91(3)  \\
        0.29(1)  &          4.72(3)  \\
        0.32(1)  &          3.92(3)  \\
        0.35(1)   &          3.59(3)  \\
        0.38(1)  &          2.62(3)  \\
        0.41(1)  &          2.60(3)  \\
        0.44(1)  &          2.09(3)  \\
        0.47(1)  &          2.27(3)  \\
    \bottomrule
  \end{tabular}
  \caption{Grandeur du champ électrique produit par une particule ponctuelle
  chargée, $E$, à différentes distances de la particule, $r$.}
  \label{tab:data_er2}
\end{table}


\section*{Présentation des graphiques}

Un graphique doit présenter les données de façon claire, concise et complète. Un
bon graphique maximise la quantité d'information fournie en minimisant l'encre
utilisée. Les axes doivent être clairement identifiés, les unités et les
incertitudes doivent être présentes. Les droites extrêmes doivent être
présentes. La légende accompagnant le graphique doit décrire son contenu
suffisamment clairement pour qu'un lecteur comprenne la nature des données qui y
sont présentées et la relation qu'on tente de mettre en évidence. Les graphiques
doivent être numérotés pour pouvoir y faire facilement référence dans le texte.

Si on reprend l'exemple de la section précédente, la figure \ref{fig:bon_er2},
montre la relation entre la grandeur du champ électrique et l'inverse du carré
de la distance. Théoriquement, cette relation devrait être linéaire et c'est
bien ce que semble montrer le graphique.

\begin{figure}
  \centering
  \includegraphics[width=0.7\textwidth]{exemple_bon_graphique_er2.pdf}
  \caption{Grandeur du champ électrique d'une particule ponctuelle chargée. La
  relation linéaire entre la grandeur du champ électrique et l'inverse du carré
  de la distance est cohérente avec l'expression théorique du champ électrique
  produit par une charge ponctuelle (équation \ref{eq:champ_ponctuel}).}
  \label{fig:bon_er2}
\end{figure}


\section*{Vérification d'un modèle}

Vérifier un modèle implique de s'assurer que les mesures concordent avec les
prédictions du modèle. Plus spécifiquement, on vérifie si
\begin{itemize}
  \item la distribution des points est cohérente avec ce que prédit le
    modèle (ils devraient être répartis aléatoirement autour de la courbe de
    régression et on doit avoir un maximum de 20\% de points singuliers);
  \item les paramètres du modèle obtenus expérimentalement sont cohérents
    avec les paramètres théoriques.
\end{itemize}
Très souvent, dans le cours, le modèle sera linéaire ou il aura été linéarisé.
Par exemple, en analysant la relation entre $E$ et $1/r^2$ plutôt que celle
entre $E$ et $r$, on a linéarisé l'expression de la grandeur du champ électrique
produit par une charge ponctuelle. Dans ce cas, les points devraient être
distribués le long d'une droite, et les paramètres de la droite (pente et
ordonnée à l'origine) devraient avoir des valeurs de $kq$ et 0, respectivement.

\section*{Cahier de laboratoire}

La tenue d'un cahier de laboratoire fait partie des bonnes pratiques en
recherche. Le cahier de laboratoire devrait être clairement identifié (nom du
propriétaire et sujet). Pour chaque expérience, le cahier de laboratoire devrait
contenir le titre de l'expérience et la date à laquelle l'expérience a été
réalisée. On met ensuite le travail préalable au laboratoire, puis les données
et observations qualitatives obtenues lors de l'expérience. Les schémas de
circuits et certains calculs simples devraient aussi se retrouver dans le
cahier de laboratoire.


\printbibliography


\end{document}



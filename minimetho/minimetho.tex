\documentclass[nofonts]{tufte-handout}

\usepackage{polyglossia}
\setdefaultlanguage{french}
\usepackage{booktabs}
\usepackage[locale=FR]{siunitx}
\usepackage{graphicx}
\usepackage{enumitem}
\usepackage{tikz}
\usepackage{amsmath}

\usepackage{fontspec}
\usepackage{ifluatex}
\setmainfont[Renderer=Basic, Numbers=OldStyle, Scale = 1.0]{TeX Gyre Pagella}
\setsansfont[Renderer=Basic, Scale=0.90]{TeX Gyre Heros}
\setmonofont[Renderer=Basic]{TeX Gyre Cursor}
\ifluatex
  \newcommand{\textls}[2][5]{%
    \begingroup\addfontfeatures{LetterSpace=#1}#2\endgroup
  }
  \renewcommand{\allcapsspacing}[1]{\textls[15]{#1}}
  \renewcommand{\smallcapsspacing}[1]{\textls[10]{#1}}
  \renewcommand{\allcaps}[1]{\textls[15]{\MakeTextUppercase{#1}}}
  \renewcommand{\smallcaps}[1]{\smallcapsspacing{\scshape\MakeTextLowercase{#1}}}
  \renewcommand{\textsc}[1]{\smallcapsspacing{\textsmallcaps{#1}}}
\fi

\sisetup{
  mode=text,
  reset-text-family=false,
  reset-text-series=false,
  per-mode=symbol,
  uncertainty-mode=separate,
}

\newcommand{\F}{\boldsymbol{\vec{F}}}
\newcommand{\vv}{\boldsymbol{\vec{v}}}
\newcommand{\va}{\boldsymbol{\vec{a}}}

\newif\ifsolution
%\solutiontrue  % Montrer les solutions dans le pdf
\solutionfalse  % Cacher les solutions


\title{Mini-guide méthodologique}
\author{Loïc Séguin-Charbonneau}
\date{203-NYB-05, Automne 2024}

\begin{document}

\maketitle

Ce mini-guide est un aide mémoire très condensé, pas un ouvrage de référence
complet. Il existe de nombreuses références qui devraient être
utilisées pour avoir des explications détaillées\cite{laflamme_2018,boisclair_page_2014}.

\section{Règles d'écriture du résultat d'un mesurage}

Lorsqu'on écrit un résultat de mesurage, que ce soit un résultat de mesurage
obtenu directement avec un instrument de mesure ou obtenu indirectement à
partir de calculs, on écrit la valeur mesurée, l'incertitude et l'unité en
respectant les règles suivantes:
\begin{enumerate}
  \item l'incertitude s'écrit avec un ou deux chiffres significatifs;
  \item le nombre de positions décimales de la valeur mesurée est le même que
    celui de l'incertitude;
  \item la valeur mesurée est suivie du signe $\pm$, puis de l'incertitude, le
    tout entre parenthèses et suivi de l'unité;
  \item lorsque nécessaire, on utilise la notation scientifique en prenant soin
    d'utiliser la même puissance de 10 pour la valeur mesurée et l'incertitude.
\end{enumerate}

\begin{table}
  \begin{tabular}{lll}
    \toprule
    Exemple correct    & Exemple incorrect  &  Règle violée  \\
    \midrule
    \qty{23.4123(18)}{\milli\ampere}  &  \qty{23.41231(187)}{\milli\ampere}  &  1  \\
    \qty{123.7(8)}{\micro\farad}  &  $\left(\num{123.7} \pm \num{0.08}\right)\unit{\micro\farad}$  &  2  \\
    \qty{8.79(3)}{\volt}  &  $\num{8.79} \pm \num{0.03} \unit{\volt}$  &  3  \\
    \qty{4.284(2)e-5}{\meter}  &  $\left(\num{4.284e-5} \pm
    \num{2e-8}\right)\unit{\meter}$  &  4\\
    \bottomrule
  \end{tabular}
  \caption{Exemples d'utilisation des règles d'écriture pour les résultats de mesurage.}
\end{table}



\section{Évaluation et propagation des incertitudes}

Lorsqu'on effectue un mesurage, il est impossible de connaître exactement la
valeur du mesurage. Il faut évaluer l'incertitude en additionnant
l'incertitude instrumentale et l'incertitude liée au contexte.
L'incertitude instrumentale peut être déterminée à partir de règles empiriques
(par exemple, prendre la moitié de la plus petite division pour une lecture sur
un instrument graduée) ou être fournie par le fabricant (par exemple, le
fabricant d'un multimètre numérique donne un tableau détaillé pour le calcul des
incertitudes instrumentales).

On évalue les incertitudes liées au contexte en tenant compte de la méthode de
mesure. Lorsqu'il fait cette évaluation, l'expérimentateur doit être confiant
que la valeur du mesurande se situe dans l'intervalle définit par l'incertitude.
Par exemple, pour un instrument à affichage numérique, si la valeur sur
l'affichage fluctue, on peut attribuer une incertitude liée au contexte de la
moitié de l'écart entre les valeurs extrêmes.

Dans des cas simples, on peut calculer l'incertitude sur un résultat de mesurage
composé en utilisant les règles résumées dans le
tableau~\ref{tab:regles_simples}. Ces règles simples surestiment généralement
l'incertitude composée, mais elles donnent le bon ordre de grandeur.

\begin{table}
  \renewcommand{\arraystretch}{1.4}
  \begin{tabular}{llll}
    \toprule
    Opération   & &  Valeur &  Incertitude  \\
    \midrule
    Somme  &  $R = A + B$
      &  $\tilde{R} = \tilde{A} + \tilde{B}$
      & $\Delta R = \Delta A + \Delta B$ \\
    Différence  &  $R = A - B$
      &  $\tilde{R} = \tilde{A} - \tilde{B}$
      & $\Delta R = \Delta A + \Delta B$ \\
    Produit  &   $R = A \times B$
      &  $\tilde{R} = \tilde{A} \times \tilde{B}$
      & $\Delta R = \left|\tilde{R}\right| \left(
          \frac{\Delta A}{\left|\tilde{A}\right|} + \frac{\Delta B}{\left|\tilde{B}\right|}
        \right)$ \\
    Quotient  &  $R = A / B$
      &  $\tilde{R} = \tilde{A} / \tilde{B}$
      & $\Delta R = \left|\tilde{R}\right| \left(
          \frac{\Delta A}{\left|\tilde{A}\right|} + \frac{\Delta B}{\left|\tilde{B}\right|}
        \right)$ \\
    Exposant  &  $R = A^n$
      &  $\tilde{R} = \tilde{A}^n$
      & $\Delta R = \left|\tilde{R}\right| \left(\left|n\right|\frac{\Delta A}{\left|\tilde{A}\right|}\right)$ \\
      \bottomrule
  \end{tabular}
  \caption{Règles simples de calcul d'incertitude pour des variables
  indépendantes $A = \tilde{A} \pm \Delta A$ et $B = \tilde{B} \pm \Delta B$.}
  \label{tab:regles_simples}
\end{table}


\section{Présentation des tableaux}

Un tableau doit présenter les données de façon claire, concise et complète. Un
bon tableau maximise la quantité d'information fournie en minimisant l'encre
utilisée. Les colonnes doivent être clairement identifiées, les unités et les
incertitudes doivent être présentes, les données doivent être écrites en
respectant les règles d'écriture. La légende accompagnant le tableau doit
décrire son contenu suffisamment clairement pour qu'un lecteur comprenne la
nature des données qui y sont présentées. Les tableaux doivent être numérotés
pour pouvoir y faire facilement référence dans le texte.


\section{Présentation des graphiques}

Un graphique doit présenter les données de façon claire, concise et complète. Un
bon graphique maximise la quantité d'information fournie en minimisant l'encre
utilisée. Les axes doivent être clairement identifiés, les unités et les
incertitudes doivent être présentes. La légende accompagnant le graphique doit
décrire son contenu suffisamment clairement pour qu'un lecteur comprenne la
nature des données qui y sont présentées et la relation qu'on tente de mettre en
évidence. Les graphiques doivent être numérotés pour pouvoir y faire facilement
référence dans le texte.


\section{Cahier de laboratoire}
\section{Vérification d'un modèle}




\bibliographystyle{plainnat}
\bibliography{../elecmag}

\end{document}



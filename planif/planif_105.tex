\documentclass[10pt]{article}

\usepackage{polyglossia}
\usepackage[landscape, top=1.5cm, left=1.5cm, right=1.5cm]{geometry}
\usepackage{longtable}
\usepackage{fontspec}
\usepackage{booktabs}
\usepackage{siunitx}

\setdefaultlanguage{french}
\setmainfont{Tex Gyre Pagella}
\setsansfont{Tex Gyre Heros}
\sisetup{
  mode=text,
  reset-text-family=false,
  reset-text-series=false,
}

\title{Planification -- Électricité et magnétisme}
\author{203-NYB-05  Groupes 1051 et 1052}
\date{Automne 2024, mis à jour le 13 septembre 2024}

\begin{document}

\makeatletter
\begin{center}
  {\LARGE \noindent\@title}
  \vspace{0.6em}

  {\large \noindent\@author}
  \vspace{0.2em}

  {\large \noindent\@date}
\end{center}
\vspace{2em}
\makeatother

\noindent Ceci est un document vivant qui sera mis à jour régulièrement au cours de la
session.

\vspace{1em}

\noindent Le manuel du cours est \emph{Physique 2: Électricité et magnétisme} par René
Lafrance, publié chez Chenelière Éducation.

\vspace{1em}

\noindent Les buts spécifiques de chaque chapitre se trouvent à la première
page du chapitre dans le manuel. Un résumé des concepts et résultats importants
se trouve à la fin du chapitre, juste avant la série d'excercices.

\vspace{1em}

\noindent Les exercices sont subdivisés en trois catégories:
\textbf{É}chauffement, \textbf{P}rincipale et e\textbf{X}tra. Le niveau de
compétence attendu correspond à la catégorie principale. Je vous recommande de
faire les exercices d'échauffement en premier. Si vous voulez un défi
supplémentaire, vous pouvez essayer les exercices de la catégorie extra, mais
ceux-ci vont au-delà du niveau qui sera évalué en examen.

\vspace{1em}

\noindent Les exercices sont identifiés par deux nombres séparés par un point.
Le premier nombre correspond au numéro du chapitre, le second au numéro de
l'exercice dans ce chapitre. Par exemple, l'exercice 2.13 est le P13 dans le
chapitre 2.

\sffamily
\renewcommand{\arraystretch}{1.2}
\begin{longtable}{cccp{6cm}lp{9cm}}
  \toprule
  \textbf{Sem.}  &  \textbf{Pér.}  &  \textbf{Date}  &  \textbf{Thème}
    &  \textbf{Travail avant le cours}
    &  \textbf{Travail après le cours}  \\
  \midrule
  \endhead
  0     &  1     &  22 août  &  Présentation du plan de cours
    &  &  \\
  0     &  2     &  22 août  &  Sprint de mécanique
    &  &  Exercices de révision de mécanique  \\
  \midrule
  1     &  1, 2  &  26 août     &  C1: Charge électrique, isolants, conducteurs
    & Lire \S 1.1, 1.2, 1.3
    & É: 1.1, 1.2 \newline 
      P: 1.4, 1.6, 1.7, 1.8 \newline 
      X: 1.5  \\
  1     &  3     &  28 août     &  C1: Loi de Coulomb
    & Lire \S 1.4
    & É: 1.10, 1.11, 1.12, 1.13, 1.15, 1.21 \newline
      P: 1.16, 1.17, 1.19, 1.23, 1.24, 1.28, 1.29 \newline 
      X: 1.26, 1.27, 1.31, 1.32, 1.33 \\
  1     &  4     &  29 août  &  Exercices C1
    &   &    \\
  1     &  5     &  29 août   &  Exemple problème intégrateur
    &   &    \\
  \midrule
  2     &  1     &  4 sept          &  C2: Champ électrique
    &  Lire \S 2.1, 2.2
    &  É: 2.1, 2.3, 2.10 \newline 
       P: 2.5, 2.6, 2.7, 2.8, 2.9, 2.13, 2.14, 2.15, 2.16, 2.18  \newline
       X: 2.4, 2.19, 2.20 \\
  2     &  2, 3     &  5 sept (lundi)  &  C2: Distribution de charge
    &  Lire \S 2.4, 2.5 (pp. 57--60)
    &  P: 2.25, 2.26, 2.28, 2.43 \newline
       X: 2.30, 2.32, 2.36, 2.41 \\
  \midrule
  3     &  1, 2  &  9 sept    &  C2: Lignes de champ, mvt dans un champ uniforme
    & Lire \S 2.6, 2.7
    &  É: 2.46, 2.49, 2.50 \newline 
       P: 2.47, 2.48, 2.51, 2.52, 2.53, 2.54, 2.55, 2.56, 2.57 \\
  3     &  3     &  11 sept  &  \textbf{Problème intégrateur A}  \\
  3     &  4     &  12 sept   &  Tableaux et graphiques
    & Lire miniguide métho
    & Exercices tableaux et graphiques  \\
  3     &  5     &  12 sept   &  Exercices C2  \\
  \midrule
  4     &  1  &  16 sept    &  C3: Théorème de Gauss
    & Lire \S 3.1, 3.2, 3.3, 3.4, 3.5 
    &  É: 3.1 \newline 
       P: 3.4, 3.6, 3.10, 3.11, 3.14, 3.23, 3.24, 3.35 \newline
       X: 3.9, 3.17, 3.30, 3.40 \\
  4     &  2  &  16 sept   &  C4: Potentiel électrique
    & Lire \S 4.1, 4.2, 4.3, 4.4
    & É: 4.1, 4.2, 4.7, 4.8, 4.9, 4.17, 4.18, 4.21  \newline 
      P: 4.4, 4.5, 4.6, 4.10, 4.11, 4.12, 4.13, 4.19, 4.20, 4.22, 4.23, 4.24, 4.25 \newline 
      X: 4.14, 4.16 \newline \\
  4     &  3     &  18 sept   &  C4: Potentiel électrique
    & Lire \S 4.5, 4.6
    & P: 4.26, 4.27, 4.28, 4.30, 4.31, 4.34, 4.35, 4.37, 4.38 \newline
      X: 4.39, 4.40 \\
  4     &  4, 5  &  19 sept   &  \textbf{Labo 1: loi d'Ohm}
    &  Prélab 1  &  Compte-rendu labo 1 \\
  \midrule
  5     &  1, 2  &  23 sept  &  C4: Potentiel électrique
    & Lire \S 4.8, 4.9
    & P: 4.51, 4.53, 4.54, 4.56  \\
  5     &  3     &  25 sept  &  \textbf{Problème intégrateur B}  \\
  5     &  4, 5  &  26 sept  &  Exercices C4
    &    &   \\
  \midrule
  6     &  1, 2  &  30 sept  &  C5: Condensateurs
    &  &  Entraînement intra  \\
  6     &  3     &  2 oct    &  C5: Condensateurs  \\
  6     &  4, 5  &  3 oct    &  \textbf{Labo 2: condensateurs}
    &  Prélab 2  &  Compte-rendu labo 2 \\
  \midrule
  7     &  1, 2  &  7 oct    &  C6: Courant électrique  \\
  7     &  3     &  9 oct    &  C6: Courant électrique  \\
  7     &  4, 5  &  11 oct   &  \textbf{Examen intra (\qty{25}{\percent})}
    &    &   \\
  \midrule
        &        &           &  Semaine de mise à jour  \\
  \midrule
  8     &  1     &  21 oct   &  Retour sur examen intra  \\
  8     &  2     &  21 oct   &  C7: Circuits CC  \\
  8     &  3     &  23 oct   &  C7: Circuits CC  &   &   \\
  8     &  4, 5  &  24 oct   &  \textbf{Labo 3: circuits simples}
    &  Prélab 3  &   \\
  \midrule
  9     &  1, 2  &  28 oct   &  C7: Circuits CC  \\
  9     &  3     &  30 oct   &  C7: Circuits CC  \\
  9     &  4, 5  &  31 oct   &  \textbf{Labo 4: pile réelle}
    &  Prélab 4  &  Compte-rendu labo 4 \\
  \midrule
  10    &  1, 2  &  4 nov    &  C8: Champ magnétique  \\*
  10    &  3     &  6 nov    &  \textbf{Problème intégrateur C} &   &    \\*
  10    &  4, 5  &  7 nov    &  Exercices C7, C8
    &    &   \\
  \midrule
  11    &  1, 2  &  11 nov   &  C8: Champ magnétique  \\
  11    &  3     &  13 nov   &  C8: Champ magnétique  \\
  11    &  4, 5  &  14 nov   &  \textbf{Labo 5: électricité domestique}
    &  Prélab 5  &    \\
  \midrule
  12    &  1, 2  &  18 nov   &  C9: Force magnétique  \\
  12    &  4     &  20 nov   &  Exercices C8, C9  &  &  \\
  12    &  5     &  21 nov   &  Pratique mesures et instruments  &  &  \\
  \midrule
  13    &  1     &  25 nov   &  C9: Force magnétique  \\
  13    &  2     &  25 nov   &  C10: Induction électromagnétique  \\
  13    &  3     &  27 nov   &  \textbf{Problème intégrateur D}  \\
  13    &  4     &  28 nov   &  Exercices C9, C10  \\
  13    &  5     &  28 nov   &  Pratique mesures et instruments  &  &  \\
  \midrule
  14    &  1, 2  &  2 déc    &  C10: Induction électromagnétique  \\
  14    &  3     &  4 déc    &  C10: Induction électromagnétique &  &  Entraînement final  \\
  14    &  4, 5  &  5 déc    &  \textbf{Examen de laboratoire (\qty{10}{\percent}})
    &    &   \\
  \midrule
  15    &  1, 2  &  9 déc    &  JR \\
  15    &  3     &  11 déc   &  Révision   \\
  15    &  4, 5  &  12 déc   &  EC
    &    &   \\
        &        & À déterminer  &  \textbf{Examen final (\qty{35}{\percent})}  \\
  \bottomrule
\end{longtable}

\end{document}


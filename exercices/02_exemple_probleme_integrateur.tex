\documentclass[nofonts]{tufte-handout}

\usepackage{polyglossia}
\setdefaultlanguage{french}
\usepackage{booktabs}
\usepackage[locale=FR]{siunitx}
\usepackage{graphicx}
\usepackage{enumitem}
\usepackage{tikz}
\usepackage{amsmath}

\usepackage{fontspec}
\usepackage{ifluatex}
\setmainfont[Renderer=Basic, Numbers=OldStyle, Scale = 1.0]{TeX Gyre Pagella}
\setsansfont[Renderer=Basic, Scale=0.90]{TeX Gyre Heros}
\setmonofont[Renderer=Basic]{TeX Gyre Cursor}
\ifluatex
  \newcommand{\textls}[2][5]{%
    \begingroup\addfontfeatures{LetterSpace=#1}#2\endgroup
  }
  \renewcommand{\allcapsspacing}[1]{\textls[15]{#1}}
  \renewcommand{\smallcapsspacing}[1]{\textls[10]{#1}}
  \renewcommand{\allcaps}[1]{\textls[15]{\MakeTextUppercase{#1}}}
  \renewcommand{\smallcaps}[1]{\smallcapsspacing{\scshape\MakeTextLowercase{#1}}}
  \renewcommand{\textsc}[1]{\smallcapsspacing{\textsmallcaps{#1}}}
\fi

\sisetup{
  mode=text,
  reset-text-family=false,
  reset-text-series=false,
  per-mode=symbol,
}

\newcommand{\F}{\boldsymbol{\vec{F}}}
\newcommand{\vv}{\boldsymbol{\vec{v}}}
\newcommand{\va}{\boldsymbol{\vec{a}}}

\newif\ifsolution
%\solutiontrue  % Montrer les solutions dans le pdf
\solutionfalse  % Cacher les solutions


\title{Exemple de problème intégrateur -- Charge électrique}
\author{Loïc Séguin-Charbonneau}
\date{203-NYB-05, Automne 2024}

\begin{document}

\maketitle

\begin{marginfigure}[10\baselineskip]
  \begin{tikzpicture}
    \draw[ultra thick] (0, 0) -- (0, 4) -- (3, 4);
    \draw[ultra thick] (0, 2) -- (0.6, 2);
    \fill (0.6, 2) circle (4pt);
    \fill (1.32, 2) circle (4pt);
    \draw (0.6, 4) -- ++(-70:2.2);
  \end{tikzpicture}
\end{marginfigure}

On considère deux balles métalliques identiques de \qty{300}{g}. Une des balles
est initialement neutre et l'autre porte une charge $q$. Les deux balles sont
mises en contact, puis on en place une sur un support fixé au mur et l'autre
suspendue à une corde attachée au plafond. On positionne les balles tel
qu'illustré ci-contre. La corde a une longueur de \qty{20}{cm} et forme un
angle de \qty{20}{\degree} avec la verticale lorsque le système est à l'équilibre.

Quelle est la charge $q$?

\end{document}


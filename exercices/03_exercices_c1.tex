\documentclass[nofonts]{tufte-handout}

\usepackage{polyglossia}
\setdefaultlanguage{french}
\usepackage{booktabs}
\usepackage[locale=FR]{siunitx}
\usepackage{graphicx}
\usepackage{enumitem}
\usepackage{tikz}
\usepackage{amsmath}
\usepackage{mhchem}

\usepackage{fontspec}
\usepackage{ifluatex}
\setmainfont[Renderer=Basic, Numbers=OldStyle, Scale = 1.0]{TeX Gyre Pagella}
\setsansfont[Renderer=Basic, Scale=0.90]{TeX Gyre Heros}
\setmonofont[Renderer=Basic]{TeX Gyre Cursor}
\ifluatex
  \newcommand{\textls}[2][5]{%
    \begingroup\addfontfeatures{LetterSpace=#1}#2\endgroup
  }
  \renewcommand{\allcapsspacing}[1]{\textls[15]{#1}}
  \renewcommand{\smallcapsspacing}[1]{\textls[10]{#1}}
  \renewcommand{\allcaps}[1]{\textls[15]{\MakeTextUppercase{#1}}}
  \renewcommand{\smallcaps}[1]{\smallcapsspacing{\scshape\MakeTextLowercase{#1}}}
  \renewcommand{\textsc}[1]{\smallcapsspacing{\textsmallcaps{#1}}}
\fi

\sisetup{
  mode=text,
  reset-text-family=false,
  reset-text-series=false,
  per-mode=symbol,
}

\newcommand{\F}{\boldsymbol{\vec{F}}}
\newcommand{\vv}{\boldsymbol{\vec{v}}}
\newcommand{\va}{\boldsymbol{\vec{a}}}

\newif\ifsolution
%\solutiontrue  % Montrer les solutions dans le pdf
\solutionfalse  % Cacher les solutions


\title{Exercices chapitre 1 -- Charge électrique}
\author{Loïc Séguin-Charbonneau}
\date{203-NYB-05, Automne 2024}

\begin{document}

\maketitle

\section{Échauffement}

\subsection{Un peigne chargé}

\marginnote{Chapitre 1, exercice 1.1.3 dans \citet{seguin_em_2010}}
Après avoir frotté un peigne dans ses cheveux afin de le charger, Béatrice
l'approche de petits morceaux de papier. Ces derniers sont d'abord attirés par
le peigne, mais dès qu'ils le touchent, ils sont repoussés. Expliquez pourquoi.

\subsection{Nouvelle particule}

\marginnote{Chapitre 1, Q1 dans \citet{benson_em_2024}}
Un journal rapporte que l'on vient de découvrir une nouvelle particule
élémentaire de charge \SI{9e-19}{\coulomb}. Quelle est votre réaction?


\subsection{Particules alpha}

À quelle distance l'une de l'autre doivent se trouver deux particules alpha\sidenote{Une
particule alpha est un noyau de \ce{^4_2He}.} pour
que la force entre elles soit d'une grandeur de \qty{10}{\newton}?


\section{Série principale}

\subsection{Symétrie}

Deux charges de \qty{4.2}{\micro\coulomb} et \qty{-4.2}{\micro\coulomb} sont
fixées à une distance de \qty{100}{\micro\meter} l'une de l'autre. On place une
troisième charge de \qty{-7}{\micro\coulomb} à une distance de
\qty{50}{\micro\meter} au-dessus du point milieu entre les deux premières
charges (voir figure ci-contre).
\begin{marginfigure}
  \begin{tikzpicture}
    \fill (-2, 0) circle (4pt) node[below, yshift=-4] {\qty{4.2}{\micro\coulomb}};
    \fill (2, 0) circle (4pt) node[below, yshift=-4] {\qty{-4.2}{\micro\coulomb}};
    \fill (0, 1) circle (4pt) node[above, yshift=4] {\qty{-7}{\micro\coulomb}};
    \draw[dashed] (-2, 0) -- (2, 0);
    \draw[dashed] (0, 0) -- (0, 1);
  \end{tikzpicture}
\end{marginfigure}


Quelle est la force sur la troisième charge?


\subsection{Sphères en contact}

On met en contact deux sphères métalliques identiques dont une est neutre et
l'autre est chargée négativement.
Puis, on les sépare. Si la force électrique sur une des sphères lorsqu'elles
sont séparées de \qty{3}{\centi\meter} est de \qty{76}{\milli\newton}, quelle
est le changement de masse qu'a subit la sphère neutre lors du contact.

\section{Extra}

\subsection{L'incroyable intensité de la force électrique}

\marginnote{Chapitre 1, exercice 1.2.20 dans \citet{seguin_em_2010}}
Dans un mauvais scénario de science-fiction, un savant fou prélève, pendant la
nuit, $N$ électrons de la Terre et les dépose sur la Lune.
\begin{enumerate}[label=\alph*)]
  \item Quelle doit être
    la valeur de $N$ pour que la force électrique entre la Terre et la Lune qui
    résulte de cet exploit soit égale à la force gravitationnelle qu'elles exercent
    l'une sur l'autre? (Les masses de la Terre et de la Lune sont respectivement de
    \qty{5.98e24}{kg} et de \qty{7.35e22}{kg}).
  \item Quelle est la masse totale de ces
    $N$ électrons?
\end{enumerate}




\bibliography{../elecmag}
\bibliographystyle{plainnat}

\end{document}


